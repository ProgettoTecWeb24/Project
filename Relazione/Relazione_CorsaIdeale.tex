\documentclass[a4paper, 12pt]{article}
\usepackage{hyperref}
\usepackage{geometry}
\usepackage{longtable}
\usepackage[table,xcdraw]{xcolor}
\usepackage{float}
\usepackage{ragged2e}
\usepackage{graphicx}  % Per inserire il logo
\usepackage{titlesec}  % Per modificare lo stile dei titoli

\geometry{
    a4paper,
    left=25mm,
    right=25mm,
    top=20mm,
    bottom=20mm,
}

\setlength{\parskip}{1em}
\setlength{\parindent}{0pt}

\graphicspath{{images}}

% Modifica lo stile dei titoli section*
\titleformat{\section*}
  {\normalfont\normalsize\bfseries\centering}  % Formato centrato

% Indice
\renewcommand{\contentsname}{Indice}

\begin{document}

% Titolo e informazioni del progetto
\title{\textbf{Relazione Progetto Tecnologie Web}}
\author{Sito Web: \textit{CorsaIdeale}}
\date{Anno Accademico 2024-2025}

\begin{center}
\maketitle

% Inserimento logo
\includegraphics[width=0.2\textwidth]{logo.png}
\end{center}

\section*{\centering Componenti del Gruppo}
\begin{center}
Marin Davide - 2068234\\
\vspace{0.2cm}
Eddy Pinarello - 2075535\\
\vspace{0.2cm}
Rubino Alfredo - 2076435\\
\vspace{0.2cm}
Salvò Giovanni - 2074010
\end{center}

\section*{\centering Informazioni Generali}
\begin{center}
\textbf{Indirizzo sito web}: \url{http://tecweb.studenti.math.unipd.it/epinarello/}\\
\vspace{0.2cm}
\textbf{Email referente gruppo}: eddy.pinarello@studenti.unipd.it
\end{center}

\section*{\centering Credenziali Utenti}
\begin{center}
\begin{longtable}{|l|l|l|}
\hline
\rowcolor[HTML]{094074}
{\color[HTML]{FFFFFF} Ruolo} & {\color[HTML]{FFFFFF} Username} & {\color[HTML]{FFFFFF} Password}\\
\hline
Utente & user & user\\
\hline
Amministratore & admin & admin\\
\hline
\end{longtable}
\end{center}

\newpage
\tableofcontents

\newpage
\begin{justify}
    

\section{Introduzione}
Il progetto \textbf{CorsaIdeale} nasce dall'idea di fornire una piattaforma online dedicata alle recensioni e ai servizi per i runner. Il sito offre informazioni sulle scarpe da corsa, analisi approfondite, strumenti per il monitoraggio delle performance e un sistema di classificazione basato sui chilometri percorsi settimanalmente. L'obiettivo principale \`e migliorare l'esperienza degli utenti, guidandoli nella scelta delle scarpe ideali.

\section{Analisi dei requisiti}
Durante la fase iniziale del progetto, abbiamo raccolto e analizzato i requisiti richiesti per il sito web. Il portale deve:

\begin{itemize}
    \item Permettere agli utenti di consultare una lista dettagliata di scarpe con immagini e descrizioni.
    \item Consentire la registrazione e il login per salvare preferenze personali.
    \item Offrire un sistema di recensioni e valutazioni con commenti.
    \item Classificare gli utenti in base ai chilometri percorsi settimanalmente, assegnando loro un ruolo specifico (es. \textit{Tartaruga}, \textit{Lepre}, \textit{Lupo}, \textit{Ghepardo}).
    \item Garantire un'esperienza accessibile e responsive su dispositivi diversi.
\end{itemize}

\section{Progettazione}
La progettazione del sito \`e stata suddivisa in tre fasi principali:

\subsection{Struttura generale}
Il sito \`e stato progettato per avere una struttura semplice e intuitiva, con una \textbf{navbar} che consente di accedere rapidamente alle sezioni principali:
\begin{itemize}
    \item Home: Introduzione e presentazione dei servizi offerti.
    \item Lista scarpe: Catalogo dettagliato.
    \item Chi siamo: Informazioni sul team e la missione.
    \item Area utente: Accesso al profilo personale.
\end{itemize}

\subsection{Database}
Il database \`e stato progettato per supportare le principali funzionalit\`a del sito. Include le seguenti tabelle:
\begin{itemize}
    \item \textbf{SCARPA}: Contiene informazioni su nome, marca, descrizione, tipo e immagine delle scarpe.
    \item \textbf{UTENTE}: Include username, email, password e ruolo.
    \item \textbf{RECENSIONE}: Associa gli utenti alle scarpe recensite, includendo voto e commento.
    \item \textbf{LIKES}: Gestisce il sistema di preferenze delle scarpe.
\end{itemize}

\subsection{Accessibilit\`a e responsive design}
Il design \`e stato sviluppato per essere compatibile con dispositivi di diverse dimensioni. Sono state applicate media query per adattare la visualizzazione su schermi piccoli e grandi.

\section{Implementazione}
\subsection{Frontend}
Il frontend \`e stato sviluppato utilizzando HTML5, CSS3 e JavaScript. Sono stati utilizzati framework e librerie moderne per garantire:
\begin{itemize}
    \item Navigazione fluida e interfaccia intuitiva.
    \item Animazioni leggere per migliorare l'esperienza utente.
    \item Struttura semantica del codice per migliorare l'accessibilit\`a.
\end{itemize}

\subsection{Backend}
Il backend \`e stato implementato utilizzando PHP per la gestione della logica server-side e MySQL per il database. Funzionalit\`a principali:
\begin{itemize}
    \item Autenticazione e autorizzazione degli utenti.
    \item Salvataggio e recupero di recensioni e preferenze.
    \item Generazione dinamica di contenuti.
\end{itemize}

\section{Test e validazione}
Il sito \`e stato sottoposto a test rigorosi per verificare:
\begin{itemize}
    \item Funzionamento delle principali funzionalit\`a.
    \item Compatibilit\`a con browser moderni (Chrome, Firefox, Edge).
    \item Conformit\`a agli standard HTML e CSS tramite validatori.
\end{itemize}

\section{Conclusioni e sviluppi futuri}
Il progetto \textbf{CorsaIdeale} ha raggiunto con successo gli obiettivi prefissati, offrendo una piattaforma utile e innovativa per i runner. In futuro, prevediamo di:
\begin{itemize}
    \item Implementare un sistema di notifiche per aggiornamenti su nuove scarpe.
    \item Aggiungere una sezione dedicata a eventi e raduni.
    \item Integrare funzionalit\`a avanzate per il monitoraggio delle performance.
\end{itemize}

\vspace{1cm}


\end{justify}
\end{document}

