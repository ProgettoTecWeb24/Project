\documentclass[a4paper, 12pt]{article}
\usepackage[utf8]{inputenc}
\usepackage{hyperref}
\usepackage{geometry}
\usepackage{longtable}
\usepackage[table,xcdraw]{xcolor}
\usepackage{graphicx}
\usepackage{ragged2e}
\geometry{
    a4paper,
    left=25mm,
    right=25mm,
    top=20mm,
    bottom=20mm
}
\setlength{\parskip}{1em}
\setlength{\parindent}{0pt}
\graphicspath{{images/}}

% Indice
\renewcommand{\contentsname}{Indice}

\begin{document}

% Pagina del titolo e sezioni
\begin{titlepage}
\begin{center}
    \vspace*{0.5cm}
    \Huge \textbf{Relazione Progetto Tecnologie Web} \\
    \vspace{0.5cm}
    \Large Anno Accademico 2024-2025\\
    \vspace{1cm}
    \Huge Sito Web: \textit{CorsaIdeale} \\
    \vspace{0.5cm} 

    \includegraphics[width=0.2\textwidth]{logo.png}
\end{center}

% Componenti del Gruppo
\section*{\centering Componenti del Gruppo}
\begin{center}
Marin Davide - 2068234\\
\vspace{0.2cm}
Eddy Pinarello - 2075535\\
\vspace{0.2cm}
Rubino Alfredo - 2076435\\
\vspace{0.2cm}
Salvò Giovanni - 2074010
\end{center}

% Informazioni Generali
\section*{\centering Informazioni Generali}
\begin{center}
\textbf{Indirizzo sito web}: \url{http://tecweb.studenti.math.unipd.it/epinarello/}\\
\vspace{0.2cm}
\textbf{Email referente gruppo}: eddy.pinarello@studenti.unipd.it
\end{center}

\renewcommand{\arraystretch}{1.5} % Aumenta l'altezza delle righe del 50%

% Credenziali Utenti
\section*{\centering Credenziali Utenti}
\begin{center}
\begin{longtable}{|l|l|l|}
\hline
\rowcolor[HTML]{094074}
{\color[HTML]{FFFFFF} Ruolo} & {\color[HTML]{FFFFFF} Username} & {\color[HTML]{FFFFFF} Password}\\
\hline
Utente & user & user\\
\hline
Amministratore & admin & admin\\
\hline
\end{longtable}
\end{center}
\end{titlepage}

\newpage
\tableofcontents

\newpage
\begin{justify}
    

\section{Introduzione}
Il progetto \textbf{CorsaIdeale} nasce dall'idea di fornire una piattaforma online dedicata alle recensioni di scarpe da corsa e ai servizi per i runner. Il sito si basa sull'interazione tra appassionati, presentando schede dettagliate dei prodotti e permettendo agli utenti di valutarli e salvare i modelli di maggior interesse. Ogni calzatura dispone di una pagina dedicata che include specifiche tecniche, recensioni con voto degli utenti registrati e il feedback di un esperto. Un aspetto distintivo del sito è il sistema di categorizzazione degli utenti, che assegna un ruolo basato sui chilometri corsi settimanalmente. Questa funzionalità, oltre a motivare gli utenti, contribuisce a creare una community più coinvolgente e orientata al miglioramento personale. Gli utenti possono creare liste personali di scarpe preferite, consultare e pubblicare recensioni, accedendo a pagine personalizzate in base al proprio profilo. L'obiettivo principale del sito è rendere piacevole l'esperienza degli utenti, guidandoli nella scelta delle scarpe ideali.


\section{Analisi dei Requisiti}

\subsection{Target di Utenza}

Il pubblico a cui si rivolge \textbf{CorsaIdeale} è molto ampio, partendo dai principianti fino ad arrivare agli atleti esperti. I runner meno esperti possono utilizzare la piattaforma per orientarsi nella scelta della prima scarpa, mentre gli utenti più avanzati possono esplorare modelli specializzati per migliorare le proprie prestazioni. Oltre alle funzionalità riservate agli utenti registrati, cioè la possibilità di recensire e salvare i modelli di scarpe, il sito prevede una dashboard amministrativa per la gestione del database e delle recensioni, accessibile esclusivamente agli amministratori.\\
L'interfaccia è progettata per essere semplice e intuitiva, garantendo l'accessibilità a tutti gli utenti con disabilità e a chi utilizza dispositivi non all'avanguardia. Allo stesso tempo, la struttura del sito supporta funzionalità avanzate per chiunque desideri interagire attivamente con la piattaforma.\\
La funzionalità di ricerca del sito è stata ideata cercando di ricoprire tutte le fasce di utenza definite dalla metafora della pesca:
\begin{itemize}
    \item \textbf{Tiro perfetto}: gli schemi organizzativi esatti consentono ad un utente deciso di raggiungere immediatamente la scarpa desiderata.
    \item \textbf{Trappola per aragoste}: gli schemi organizzativi ambigui permettono all'utente di approfondire le conoscenze grazie alla possibilità di suddividere la lista in sottocategorie.
    \item \textbf{Pesca con la rete}: il sito è navigabile con facilità anche da un utente che non vuole perdersi nulla, grazie alla divisione in macrosezioni.
    \item \textbf{Boa di segnalazione}: la funzionalità che permette di salvare le scarpe preferite è perfetta per chi vuole ritrovarle in un secondo momento.
\end{itemize}

\subsection{Funzionalità del Sito}

Oltre a quelle elencate, ogni categoria di utente ha a disposizione tutte le funzionalità di quella precedente. Segue l'elenco delle funzionalità alle quali ha accesso ogni tipologia di utente:

\begin{itemize}
    \item \textbf{Utente generico}:
        \begin{itemize}
            \item Visualizzazione della Lista Scarpe;
            \item Visualizzazione delle informazioni della singola scarpa;
            \item Visualizzazione delle recensioni degli utenti registrati;
            \item Registrazione e assegnazione del ruolo.
        \end{itemize}
    \item \textbf{Utente registrato}:
        \begin{itemize}
            \item Login;
            \item Recensioni con voto;
            \item Salvataggio delle scarpe;
            \item Gestione e modifica delle proprie recensioni tramite pagina personale.
        \end{itemize}
    \item \textbf{Amministratore}:
        \begin{itemize}
            \item Accesso a dashboard amministrativa;
            \item Gestione e modifica delle recensioni di tutti gli utenti;
            \item Gestione e modifica del database;
            \item boh
        \end{itemize}
\end{itemize}


\section{Progettazione}

\subsection{Struttura generale}

Il sito si articola in quattro sezioni principali, ognuna con funzionalità specifiche. La \textbf{home page} offre una panoramica delle ultime scarpe aggiunte e delle funzionalità del sito, fornendo un'introduzione chiara e accattivante. La \textbf{lista scarpe} consente di esplorare l'intero catalogo disponibile, con la possibilità di filtrare i risultati in base a caratteristiche specifiche come marca, tipo di utilizzo o valutazione media.

Ogni scarpa è dotata di una pagina dedicata, dove gli utenti registrati possono lasciare recensioni e assegnare voti. Queste pagine mostrano informazioni dettagliate sui modelli, incluse descrizioni tecniche e immagini. Gli utenti possono anche salvare le scarpe preferite tramite un sistema di "like". Il profilo personale consente di visualizzare le scarpe salvate, le recensioni lasciate e il proprio ruolo, che viene aggiornato automaticamente in base ai chilometri settimanali corsi. Il sistema di ruoli, articolato in categorie come \textit{Tartaruga}, \textit{Lepre}, \textit{Lupo} e \textit{Ghepardo}, motiva gli utenti a migliorare le proprie prestazioni.

Per garantire un ambiente sicuro e moderato, gli amministratori possono gestire il database delle scarpe e moderare le recensioni, rimuovendo contenuti inappropriati. Questa gestione avviene tramite una dashboard dedicata, accessibile esclusivamente a utenti con privilegi amministrativi.

Il sito \`e stato progettato per avere una struttura semplice e intuitiva, con una \textbf{navbar} che consente di accedere rapidamente alle sezioni principali:
\begin{itemize}
    \item Home: Introduzione e presentazione dei servizi offerti.
    \item Lista scarpe: Catalogo dettagliato.
    \item Chi siamo: Informazioni sul team e la missione.
    \item Area utente: Accesso al profilo personale.
\end{itemize}

\subsection{Database}
Il database \`e stato progettato per supportare le principali funzionalit\`a del sito. Include le seguenti tabelle:
\begin{itemize}
    \item \textbf{SCARPA}: Contiene informazioni su nome, marca, descrizione, tipo e immagine delle scarpe.
    \item \textbf{UTENTE}: Include username, email, password e ruolo.
    \item \textbf{RECENSIONE}: Associa gli utenti alle scarpe recensite, includendo voto e commento.
    \item \textbf{LIKES}: Gestisce il sistema di preferenze delle scarpe.
\end{itemize}

\subsection{Accessibilit\`a e responsive design}
Il design \`e stato sviluppato per essere compatibile con dispositivi di diverse dimensioni. Sono state applicate media query per adattare la visualizzazione su schermi piccoli e grandi.

\section{Implementazione}
\subsection{Frontend}
Il frontend \`e stato sviluppato utilizzando HTML5, CSS3 e JavaScript. Sono stati utilizzati framework e librerie moderne per garantire:
\begin{itemize}
    \item Navigazione fluida e interfaccia intuitiva.
    \item Animazioni leggere per migliorare l'esperienza utente.
    \item Struttura semantica del codice per migliorare l'accessibilit\`a.
\end{itemize}

\subsection{Backend}
Il backend \`e stato implementato utilizzando PHP per la gestione della logica server-side e MySQL per il database. Funzionalit\`a principali:
\begin{itemize}
    \item Autenticazione e autorizzazione degli utenti.
    \item Salvataggio e recupero di recensioni e preferenze.
    \item Generazione dinamica di contenuti.
\end{itemize}

\section{Test e validazione}
Il sito \`e stato sottoposto a test rigorosi per verificare:
\begin{itemize}
    \item Funzionamento delle principali funzionalit\`a.
    \item Compatibilit\`a con browser moderni (Chrome, Firefox, Edge).
    \item Conformit\`a agli standard HTML e CSS tramite validatori.
\end{itemize}

\section{Conclusioni e sviluppi futuri}
Il progetto \textbf{CorsaIdeale} ha raggiunto con successo gli obiettivi prefissati, offrendo una piattaforma utile e innovativa per i runner. In futuro, prevediamo di:
\begin{itemize}
    \item Implementare un sistema di notifiche per aggiornamenti su nuove scarpe.
    \item Aggiungere una sezione dedicata a eventi e raduni.
    \item Integrare funzionalit\`a avanzate per il monitoraggio delle performance.
\end{itemize}

\vspace{1cm}


\end{justify}
\end{document}

