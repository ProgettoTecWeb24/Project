\documentclass[a4paper, 12pt]{article}
\usepackage[utf8]{inputenc}
\usepackage{hyperref}
\usepackage{geometry}
\usepackage{longtable}
\usepackage[table,xcdraw]{xcolor}
\usepackage{graphicx}
\usepackage{ragged2e}
\usepackage[T1]{fontenc}
\geometry{
    a4paper,
    left=25mm,
    right=25mm,
    top=20mm,
    bottom=20mm
}
\setlength{\parskip}{1em}
\setlength{\parindent}{0pt}
\graphicspath{{images/}}

% Indice
\renewcommand{\contentsname}{Indice}

\begin{document}

% Pagina del titolo e sezioni
\begin{titlepage}
\begin{center}
    \vspace*{0.5cm}
    \Huge \textbf{Relazione Progetto Tecnologie Web} \\
    \vspace{0.5cm}
    \Large Anno Accademico 2024-2025\\
    \vspace{1cm}
    \Huge Sito Web: \textit{CorsaIdeale} \\
    \vspace{0.5cm} 

    \includegraphics[width=0.2\textwidth]{logo.png}
\end{center}

% Componenti del Gruppo
\section*{\centering Componenti del Gruppo}
\begin{center}
Marin Davide - 2068234\\
\vspace{0.2cm}
Pinarello Eddy - 2075535\\
\vspace{0.2cm}
Rubino Alfredo - 2076435\\
\vspace{0.2cm}
Salvò Giovanni - 2074010
\end{center}

% Informazioni Generali
\section*{\centering Informazioni Generali}
\begin{center}
\textbf{Indirizzo sito web}: \url{http://tecweb.studenti.math.unipd.it/epinarello/}\\
\vspace{0.2cm}
\textbf{Email referente gruppo}: eddy.pinarello@studenti.unipd.it
\end{center}

\renewcommand{\arraystretch}{1.5} % Aumenta l'altezza delle righe del 50%

% Credenziali Utenti
\section*{\centering Credenziali Utenti}
\begin{center}
\begin{longtable}{|l|l|l|}
\hline
\rowcolor[HTML]{094074}
{\color[HTML]{FFFFFF} Ruolo} & {\color[HTML]{FFFFFF} Username} & {\color[HTML]{FFFFFF} Password}\\
\hline
Utente & user & user\\
\hline
Amministratore & admin & admin\\
\hline
\end{longtable}
\end{center}
\end{titlepage}

\newpage
\tableofcontents

\newpage
\begin{justify}
    

\section{Introduzione}

Il progetto \textbf{CorsaIdeale} nasce dall'idea di fornire una piattaforma online dedicata alle recensioni di scarpe da corsa e ai servizi per i runner. Il sito si basa sull'interazione tra appassionati, presentando schede dettagliate dei prodotti e permettendo agli utenti di valutarli e salvare i modelli di maggior interesse.\\
Ogni calzatura dispone di una pagina dedicata che include specifiche tecniche, recensioni con voto degli utenti registrati e il feedback di un esperto.\\
Un aspetto distintivo del sito è il sistema di categorizzazione degli utenti, che assegna un ruolo basato sui chilometri corsi settimanalmente. Questa funzionalità, oltre a motivare gli utenti, contribuisce a creare una community più coinvolgente e orientata al miglioramento personale.\\ 
Gli utenti possono creare liste personali di scarpe preferite, consultare e pubblicare recensioni, accedendo a pagine personalizzate in base al proprio profilo. L'obiettivo principale del sito è rendere piacevole l'esperienza degli utenti, guidandoli nella scelta delle scarpe ideali.


\section{Analisi dei Requisiti}

\subsection{Target di Utenza}

Il pubblico a cui si rivolge \textbf{CorsaIdeale} è molto ampio, partendo dai principianti fino ad arrivare agli atleti esperti. I runner meno esperti possono utilizzare la piattaforma per orientarsi nella scelta della prima scarpa, mentre gli utenti più avanzati possono esplorare modelli specializzati per migliorare le proprie prestazioni. Oltre alle funzionalità riservate agli utenti registrati, cioè la possibilità di recensire e salvare i modelli di scarpe, il sito prevede una dashboard amministrativa per la gestione del database e delle recensioni, accessibile esclusivamente agli amministratori.\\
L'interfaccia è progettata per essere semplice e intuitiva, garantendo l'accessibilità a tutti gli utenti con disabilità e a chi utilizza dispositivi non all'avanguardia. Allo stesso tempo, la struttura del sito supporta funzionalità avanzate per chiunque desideri interagire attivamente con la piattaforma.\\
La funzionalità di ricerca del sito è stata ideata cercando di ricoprire tutte le fasce di utenza definite dalla metafora della pesca:
\begin{itemize}
    \item \textbf{Tiro perfetto}: gli schemi organizzativi esatti consentono ad un utente deciso di raggiungere immediatamente la scarpa desiderata.
    \item \textbf{Trappola per aragoste}: gli schemi organizzativi ambigui permettono all'utente di approfondire le conoscenze grazie alla possibilità di suddividere la lista in sottocategorie.
    \item \textbf{Pesca con la rete}: il sito è navigabile con facilità anche da un utente che non vuole perdersi nulla, grazie alla divisione in macrosezioni.
    \item \textbf{Boa di segnalazione}: la funzionalità che permette di salvare le scarpe preferite è perfetta per chi vuole ritrovarle in un secondo momento.
\end{itemize}

\subsection{Funzionalità del Sito}

Sono state implementate tre diverse tipologie di utente: utente generico, utente loggato e amministratore. Ogni categoria di utente ha a disposizione le funzionalità elencate, oltre a ereditare tutte quelle della categoria precedente.
\begin{itemize}
    \item \textbf{Utente generico}: un qualsiasi visitatore occasionale del sito, oppure un utente già registrato ma che non ha ancora effettuato il login.
        \begin{itemize}
            \item Visualizzazione della pagina Home;
            \item Visualizzazione della pagina Lista Scarpe;
            \item Visualizzazione della pagina Chi Siamo;
            \item Visualizzazione delle informazioni della singola scarpa;
            \item Visualizzazione delle recensioni degli utenti registrati;
            \item Registrazione e assegnazione del ruolo;
            \item Login.
        \end{itemize}
    \item \textbf{Utente registrato}: un utente che ha effettuato l'accesso con le sue credenziali e ha quindi accesso alla sua pagina personale.
        \begin{itemize}
            \item Recensioni con voto;
            \item Salvataggio delle scarpe;
            \item Gestione e modifica delle proprie recensioni tramite pagina personale;
            \item Modifica dello username;
            \item Modifica del ruolo assegnato.
        \end{itemize}
    \item \textbf{Amministratore}: l'utente che gestisce il sito, ha la possibilità di moderare le recensioni e inserire o modificare le scarpe all'interno del database.
        \begin{itemize}
            \item Accesso a dashboard amministrativa;
            \item Gestione e modifica delle recensioni di tutti gli utenti;
            \item Gestione e modifica del database;
            \item boh altro??
        \end{itemize}
\end{itemize}


\section{Progettazione del sito}

\subsection{Struttura del sito}

La progettazione del sito web segue un'organizzazione strutturata per garantire un accesso rapido e intuitivo alle informazioni e ai servizi offerti. La struttura è gerarchica ma progettata per mantenere un equilibrio ottimale tra ampiezza e profondità. La profondità, ossia il numero di clic necessari per navigare dalla homepage a una pagina foglia, corrisponde a INSERIRE clic, che non supera il valore ottimale di 5. Per quanto riguarda l'ampiezza, il menù presenta non più di 4-5 opzioni, restando quindi al di sotto del valore ottimale, cioè 7.\\
Questo approccio permette agli utenti di trovare rapidamente le informazioni desiderate senza subire gli effetti di un sovraccarico cognitivo o perdendosi in strutture troppo complesse.

\subsection{Composizione}

La composizione di una pagina tipo del sito è strutturata in quattro elementi principali: header, breadcrumb, contenuto e footer. Di seguito viene descritta ciascuna parte.

\subsubsection{Header}

L'header è una parte fondamentale del sito e viene aggiunto dinamicamente in tutte le pagine. Contiene:
\begin{itemize}
    \item \textbf{Logo}: situato nell'angolo in alto a sinistra e cliccabile per tornare alla homepage.
    \item \textbf{Titolo}: il nome del sito scritto in un font particolare e con i colori che riprendono quelli del logo.
    \item \textbf{Navbar}: tramite un menù a lista che rimanda alle altre pagine del sito, consente una navigazione semplice e intuitiva. Nella visualizzazione da tablet e da mobile la lista viene visualizzata tramite un \textit{hamburger menu} per esigenze di spazio.\\
    La navbar cambia leggermente in base al tipo di utente:
        \begin{itemize}
            \item \textbf{Utente generico}: visualizza le seguenti voci: "Home", "Lista Scarpe", "Chi siamo", "Accedi" e "Registrati".
            \item \textbf{Utente loggato}: al posto delle voci "Accedi" e "Registrati" dell'utente loggato, visualizza un'unica voce denominata "Il mio profilo".
            \item \textbf{Amministratore}: presenta le voci dell'utente loggato, e in aggiunta la "Dashboard Admin" per accedere alle funzionalità di gestione.
        \end{itemize}
\end{itemize}

\subsubsection{Breadcrumb}

Il breadcrumb è posizionato immediatamente sotto la navbar ed è fondamentale per aiutare l'utente a orientarsi all'interno del sito, sia da \textit{desktop} che da \textit{mobile}. Mostra il percorso gerarchico fino alla pagina corrente e permette un rapido accesso alle sezioni precedenti tramite link diretto. Per evitare i link circolari, la pagina attuale non è un link attivo.\\ 
Ad esempio, per la pagina singola di un modello di scarpe, il breadcrumb potrebbe essere: \textit{Home > Lista Scarpe > Modello Scarpa}.

\subsubsection{Contenuto}

Il contenuto varia ovviamente a seconda della singola pagina. È stato progettato mettendo nella zona sicura (\textit{above the fold}) le informazioni fondamentali per dare una risposta alle tre domande che riguardano l'orientamento:
\begin{enumerate}
    \item \textbf{Dove sono?} Questa domanda trova risposta nel titolo della pagina oppure nel breadcrumb.
    \item \textbf{Dove posso andare?} Le opzioni disponibili sono chiaramente visibili nel menu o attraverso link contestuali nel contenuto.
    \item \textbf{Di cosa si tratta?} Il contenuto principale della pagina fornisce informazioni dettagliate sull'argomento specifico tramite un titolo.
\end{enumerate}

\subsubsection{Footer}

Il footer, come l'header, è aggiunto dinamicamente a tutte le pagine del sito e contiene:
\begin{itemize}
    \item \textbf{Logo}: situato nell'angolo in basso a sinistra, stavolta non cliccabile.
    \item \textbf{Informazioni di contatto}: sono stati inseriti dei dati fittizi come un numero di telefono e un indirizzo email per contattare l'azienda, oltre all'indirizzo della sede principale (aula LUM250).
    \item \textbf{Social network}: i link ai profili per seguire l'azienda.
    \item \textbf{Badge}: icone di validazione del codice HTML e CSS.
\end{itemize}

\subsection{Descrizione delle pagine}

Il contenuto presente nelle diverse pagine che compongono il sito è organizzato come segue:
\begin{itemize}
    \item \textbf{Homepage (index.html)}:
        \begin{itemize}
            \item Contiene una sezione dedicata alle ultime uscite, con immagini delle scarpe, dettagli come marca, modello e tipo.
            \item Presenta un messaggio di benvenuto che introduce gli utenti ai servizi offerti.
            \item Descrive la meccanica dei ruoli assegnati in base ai chilometri percorsi settimanalmente, utilizzando un'efficace CTA (\textit{call to action}).
            \item Include link al resto del sito tramite un menu di navigazione principale.
        \end{itemize}
    \item \textbf{Lista Scarpe (lista.html)}:
        \begin{itemize}
            \item Pagina che elenca tutte le scarpe disponibili in un formato a griglia.
            \item Consente di riordinare le scarpe, applicare dei filtri o fare una ricerca diretta.
            \item Ogni elemento è una card che include immagine, dettagli principali (marca, modello, tipo), 
            \item Cliccando sopra la card si accede alla pagina di dettaglio della scarpa.
        \end{itemize}
    \item \textbf{Pagina Singola (paginaSingola.html)}:
        \begin{itemize}
            \item Mostra la descrizione e i dettagli completi di una scarpa selezionata.
            \item Include la recensione con il feedback di un esperto.
            \item Da questa pagina, per ogni scarpa è possibile visualizzare le recensioni degli altri utenti, oltre a poter aggiungere e modificare la propria.
        \end{itemize}
    \item \textbf{Chi Siamo (chisiamo.html)}:
        \begin{itemize}
            \item Sezione dedicata alla descrizione del progetto e agli obiettivi del sito.
            \item Presenta una foto e breve descrizione dei membri del team fondatore.
        \end{itemize}
    \item \textbf{Profilo Utente (profilo.html)}:\\ 
    Pagina personale accessibile solo agli utenti registrati, che consente di visualizzare e modificare le informazioni personali, come username e ruolo, ed effettuare il logout. Inoltre, offre accesso ad altre due pagine:
        \begin{itemize}
            \item \textbf{Recensioni Personali (profiloRecensioni.html)}: Interfaccia per gestire le recensioni rilasciate.
            \item \textbf{Scarpe Salvate (profiloScarpe.html)}: Lista per visualizzare e gestire le scarpe segnate come preferite.
        \end{itemize}
    \item \textbf{Admin Page (adminpage.html)}:\\ 
    Dashboard centrale per i soli amministratori, con accesso alle seguenti funzionalità:
        \begin{itemize}
            \item \textbf{Aggiunta Scarpe (adminAggiungiScarpa.html)}: Form per aggiungere nuove scarpe al catalogo.
            \item \textbf{Modifica Lista Scarpe (adminModificaLista.html)}: Interfaccia per aggiornare o eliminare scarpe dalla lista generale.
            \item \textbf{Modifica Singola Scarpa (adminModificaScarpa.html)}: Accessibile dalla pagina precedente, è una funzione dedicata all'aggiornamento di dettagli specifici di una scarpa.
            \item \textbf{Gestione Recensioni (adminRecensioni.html)}: Strumenti per moderare ed eventualmente eliminare le recensioni degli utenti.
        \end{itemize}
    \item \textbf{Pagine di Errore}:\\
    Vengono mostrate in caso di errori di navigazione o di server; sono state sviluppate ponendo particolare attenzione all'emotional design e presentano un titolo e un'immagine rassicuranti per l'utente.
        \begin{itemize}
            \item \textbf{Errore 404 (error404.html)}: Pagina mostrata quando una risorsa non viene trovata.
            \item \textbf{Errore 500 (error500.html)}: Pagina per gestire errori del server, con messaggi informativi per l'utente.
        \end{itemize}
    \item \textbf{Accesso e Registrazione (accedi.html e registrati.html)}:
        \begin{itemize}
            \item Pagine per la gestione degli account utente.
            \item \textbf{Accesso}: Form per l'autenticazione degli utenti registrati.
            \item \textbf{Registrazione}: Form per creare un nuovo account.
        \end{itemize}
\end{itemize}

\subsection{Suddivisione del lavoro}
Per facilitare la gestione del progetto, il lavoro è stato suddiviso tra i membri del gruppo nel seguente modo:
\begin{itemize}
    \item \textbf{Davide Marin}: 
        \begin{itemize}
            \item Pagine per la gestione degli account utente.
            \item
            \item
            \item 
        \end{itemize}
    \item \textbf{Eddy Pinarello}:
        \begin{itemize}
            \item Pagine per la gestione degli account utente.
            \item
            \item
            \item
        \end{itemize}
    \item \textbf{Alfredo Rubino}:
        \begin{itemize}
            \item Pagine per la gestione degli account utente.
            \item
            \item
        \end{itemize}
    \item \textbf{Giovanni Salvò}:
        \begin{itemize}
            \item Pagine per la gestione degli account utente.
            \item
            \item
        \end{itemize}
\end{itemize}


\section{Implementazione}

\subsection{Frontend}

Il frontend \`e stato sviluppato utilizzando HTML5, CSS3 e JavaScript. Sono stati utilizzati framework e librerie moderne per garantire:
\begin{itemize}
    \item Navigazione fluida e interfaccia intuitiva.
    \item Animazioni leggere per migliorare l'esperienza utente.
    \item Struttura semantica del codice per migliorare l'accessibilit\`a.
\end{itemize}

\subsection{Backend}
Il backend \`e stato implementato utilizzando PHP per la gestione della logica server-side e MySQL per il database. Funzionalit\`a principali:
\begin{itemize}
    \item Autenticazione e autorizzazione degli utenti.
    \item Salvataggio e recupero di recensioni e preferenze.
    \item Generazione dinamica di contenuti.
\end{itemize}

\section{Test e validazione}
Il sito \`e stato sottoposto a test rigorosi per verificare:
\begin{itemize}
    \item Funzionamento delle principali funzionalit\`a.
    \item Compatibilit\`a con browser moderni (Chrome, Firefox, Edge).
    \item Conformit\`a agli standard HTML e CSS tramite validatori.
\end{itemize}

\section{Conclusioni e sviluppi futuri}
Il progetto \textbf{CorsaIdeale} ha raggiunto con successo gli obiettivi prefissati, offrendo una piattaforma utile e innovativa per i runner. In futuro, prevediamo di:
\begin{itemize}
    \item Implementare un sistema di notifiche per aggiornamenti su nuove scarpe.
    \item Aggiungere una sezione dedicata a eventi e raduni.
    \item Integrare funzionalit\`a avanzate per il monitoraggio delle performance.
\end{itemize}

\vspace{1cm}


\end{justify}
\end{document}

